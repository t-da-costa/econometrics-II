% Options for packages loaded elsewhere
\PassOptionsToPackage{unicode}{hyperref}
\PassOptionsToPackage{hyphens}{url}
%
\documentclass[
]{article}
\usepackage{amsmath,amssymb}
\usepackage{iftex}
\ifPDFTeX
  \usepackage[T1]{fontenc}
  \usepackage[utf8]{inputenc}
  \usepackage{textcomp} % provide euro and other symbols
\else % if luatex or xetex
  \usepackage{unicode-math} % this also loads fontspec
  \defaultfontfeatures{Scale=MatchLowercase}
  \defaultfontfeatures[\rmfamily]{Ligatures=TeX,Scale=1}
\fi
\usepackage{lmodern}
\ifPDFTeX\else
  % xetex/luatex font selection
\fi
% Use upquote if available, for straight quotes in verbatim environments
\IfFileExists{upquote.sty}{\usepackage{upquote}}{}
\IfFileExists{microtype.sty}{% use microtype if available
  \usepackage[]{microtype}
  \UseMicrotypeSet[protrusion]{basicmath} % disable protrusion for tt fonts
}{}
\makeatletter
\@ifundefined{KOMAClassName}{% if non-KOMA class
  \IfFileExists{parskip.sty}{%
    \usepackage{parskip}
  }{% else
    \setlength{\parindent}{0pt}
    \setlength{\parskip}{6pt plus 2pt minus 1pt}}
}{% if KOMA class
  \KOMAoptions{parskip=half}}
\makeatother
\usepackage{xcolor}
\usepackage[margin=1in]{geometry}
\usepackage{color}
\usepackage{fancyvrb}
\newcommand{\VerbBar}{|}
\newcommand{\VERB}{\Verb[commandchars=\\\{\}]}
\DefineVerbatimEnvironment{Highlighting}{Verbatim}{commandchars=\\\{\}}
% Add ',fontsize=\small' for more characters per line
\usepackage{framed}
\definecolor{shadecolor}{RGB}{248,248,248}
\newenvironment{Shaded}{\begin{snugshade}}{\end{snugshade}}
\newcommand{\AlertTok}[1]{\textcolor[rgb]{0.94,0.16,0.16}{#1}}
\newcommand{\AnnotationTok}[1]{\textcolor[rgb]{0.56,0.35,0.01}{\textbf{\textit{#1}}}}
\newcommand{\AttributeTok}[1]{\textcolor[rgb]{0.13,0.29,0.53}{#1}}
\newcommand{\BaseNTok}[1]{\textcolor[rgb]{0.00,0.00,0.81}{#1}}
\newcommand{\BuiltInTok}[1]{#1}
\newcommand{\CharTok}[1]{\textcolor[rgb]{0.31,0.60,0.02}{#1}}
\newcommand{\CommentTok}[1]{\textcolor[rgb]{0.56,0.35,0.01}{\textit{#1}}}
\newcommand{\CommentVarTok}[1]{\textcolor[rgb]{0.56,0.35,0.01}{\textbf{\textit{#1}}}}
\newcommand{\ConstantTok}[1]{\textcolor[rgb]{0.56,0.35,0.01}{#1}}
\newcommand{\ControlFlowTok}[1]{\textcolor[rgb]{0.13,0.29,0.53}{\textbf{#1}}}
\newcommand{\DataTypeTok}[1]{\textcolor[rgb]{0.13,0.29,0.53}{#1}}
\newcommand{\DecValTok}[1]{\textcolor[rgb]{0.00,0.00,0.81}{#1}}
\newcommand{\DocumentationTok}[1]{\textcolor[rgb]{0.56,0.35,0.01}{\textbf{\textit{#1}}}}
\newcommand{\ErrorTok}[1]{\textcolor[rgb]{0.64,0.00,0.00}{\textbf{#1}}}
\newcommand{\ExtensionTok}[1]{#1}
\newcommand{\FloatTok}[1]{\textcolor[rgb]{0.00,0.00,0.81}{#1}}
\newcommand{\FunctionTok}[1]{\textcolor[rgb]{0.13,0.29,0.53}{\textbf{#1}}}
\newcommand{\ImportTok}[1]{#1}
\newcommand{\InformationTok}[1]{\textcolor[rgb]{0.56,0.35,0.01}{\textbf{\textit{#1}}}}
\newcommand{\KeywordTok}[1]{\textcolor[rgb]{0.13,0.29,0.53}{\textbf{#1}}}
\newcommand{\NormalTok}[1]{#1}
\newcommand{\OperatorTok}[1]{\textcolor[rgb]{0.81,0.36,0.00}{\textbf{#1}}}
\newcommand{\OtherTok}[1]{\textcolor[rgb]{0.56,0.35,0.01}{#1}}
\newcommand{\PreprocessorTok}[1]{\textcolor[rgb]{0.56,0.35,0.01}{\textit{#1}}}
\newcommand{\RegionMarkerTok}[1]{#1}
\newcommand{\SpecialCharTok}[1]{\textcolor[rgb]{0.81,0.36,0.00}{\textbf{#1}}}
\newcommand{\SpecialStringTok}[1]{\textcolor[rgb]{0.31,0.60,0.02}{#1}}
\newcommand{\StringTok}[1]{\textcolor[rgb]{0.31,0.60,0.02}{#1}}
\newcommand{\VariableTok}[1]{\textcolor[rgb]{0.00,0.00,0.00}{#1}}
\newcommand{\VerbatimStringTok}[1]{\textcolor[rgb]{0.31,0.60,0.02}{#1}}
\newcommand{\WarningTok}[1]{\textcolor[rgb]{0.56,0.35,0.01}{\textbf{\textit{#1}}}}
\usepackage{graphicx}
\makeatletter
\newsavebox\pandoc@box
\newcommand*\pandocbounded[1]{% scales image to fit in text height/width
  \sbox\pandoc@box{#1}%
  \Gscale@div\@tempa{\textheight}{\dimexpr\ht\pandoc@box+\dp\pandoc@box\relax}%
  \Gscale@div\@tempb{\linewidth}{\wd\pandoc@box}%
  \ifdim\@tempb\p@<\@tempa\p@\let\@tempa\@tempb\fi% select the smaller of both
  \ifdim\@tempa\p@<\p@\scalebox{\@tempa}{\usebox\pandoc@box}%
  \else\usebox{\pandoc@box}%
  \fi%
}
% Set default figure placement to htbp
\def\fps@figure{htbp}
\makeatother
\setlength{\emergencystretch}{3em} % prevent overfull lines
\providecommand{\tightlist}{%
  \setlength{\itemsep}{0pt}\setlength{\parskip}{0pt}}
\setcounter{secnumdepth}{-\maxdimen} % remove section numbering
\usepackage{bookmark}
\IfFileExists{xurl.sty}{\usepackage{xurl}}{} % add URL line breaks if available
\urlstyle{same}
\hypersetup{
  pdftitle={Econometry new},
  hidelinks,
  pdfcreator={LaTeX via pandoc}}

\title{Econometry new}
\author{}
\date{\vspace{-2.5em}2025-01-25}

\begin{document}
\maketitle

\subsection{Importation des données}\label{importation-des-donnuxe9es}

\begin{Shaded}
\begin{Highlighting}[]
\CommentTok{\#importation des différentes librairies nécessaires pour la suite du projet}
\FunctionTok{options}\NormalTok{(}\AttributeTok{repos =} \FunctionTok{c}\NormalTok{(}\AttributeTok{CRAN =} \StringTok{"https://cloud.r{-}project.org"}\NormalTok{))}
\FunctionTok{install.packages}\NormalTok{(}\FunctionTok{c}\NormalTok{(}\StringTok{"caret"}\NormalTok{, }\StringTok{"skedastic"}\NormalTok{, }\StringTok{"nortest"}\NormalTok{, }\StringTok{"olsrr"}\NormalTok{))}
\end{Highlighting}
\end{Shaded}

\begin{verbatim}
## Installing packages into '/Users/orlando/Library/R/arm64/4.3/library'
## (as 'lib' is unspecified)
\end{verbatim}

\begin{verbatim}
## 
## The downloaded binary packages are in
##  /var/folders/jq/nw67h8bx4pq6859hfrxslj7w0000gn/T//RtmpJ2Hnba/downloaded_packages
\end{verbatim}

\begin{Shaded}
\begin{Highlighting}[]
\FunctionTok{library}\NormalTok{(ggplot2)}
\FunctionTok{library}\NormalTok{(cowplot)}
\FunctionTok{library}\NormalTok{(car)}
\end{Highlighting}
\end{Shaded}

\begin{verbatim}
## Loading required package: carData
\end{verbatim}

\begin{Shaded}
\begin{Highlighting}[]
\FunctionTok{library}\NormalTok{(carData)}
\FunctionTok{library}\NormalTok{(caret)}
\end{Highlighting}
\end{Shaded}

\begin{verbatim}
## Warning: package 'caret' was built under R version 4.3.3
\end{verbatim}

\begin{verbatim}
## Loading required package: lattice
\end{verbatim}

\begin{Shaded}
\begin{Highlighting}[]
\FunctionTok{library}\NormalTok{(FactoMineR)}
\FunctionTok{library}\NormalTok{(readxl)}
\FunctionTok{library}\NormalTok{(dplyr)}
\end{Highlighting}
\end{Shaded}

\begin{verbatim}
## 
## Attaching package: 'dplyr'
\end{verbatim}

\begin{verbatim}
## The following object is masked from 'package:car':
## 
##     recode
\end{verbatim}

\begin{verbatim}
## The following objects are masked from 'package:stats':
## 
##     filter, lag
\end{verbatim}

\begin{verbatim}
## The following objects are masked from 'package:base':
## 
##     intersect, setdiff, setequal, union
\end{verbatim}

\begin{Shaded}
\begin{Highlighting}[]
\FunctionTok{library}\NormalTok{(tidyr)}
\FunctionTok{library}\NormalTok{(lmtest) }\CommentTok{\#pour Breusch{-}Pagan}
\end{Highlighting}
\end{Shaded}

\begin{verbatim}
## Loading required package: zoo
\end{verbatim}

\begin{verbatim}
## 
## Attaching package: 'zoo'
\end{verbatim}

\begin{verbatim}
## The following objects are masked from 'package:base':
## 
##     as.Date, as.Date.numeric
\end{verbatim}

\begin{Shaded}
\begin{Highlighting}[]
\FunctionTok{library}\NormalTok{(skedastic) }\CommentTok{\#pour White}
\FunctionTok{library}\NormalTok{(nortest) }\CommentTok{\#pour Anderson{-}Darling}
\end{Highlighting}
\end{Shaded}

\begin{verbatim}
## Warning: package 'nortest' was built under R version 4.3.3
\end{verbatim}

\begin{Shaded}
\begin{Highlighting}[]
\FunctionTok{library}\NormalTok{(olsrr) }\CommentTok{\#pour White}
\end{Highlighting}
\end{Shaded}

\begin{verbatim}
## Warning: package 'olsrr' was built under R version 4.3.3
\end{verbatim}

\begin{verbatim}
## 
## Attaching package: 'olsrr'
\end{verbatim}

\begin{verbatim}
## The following object is masked from 'package:datasets':
## 
##     rivers
\end{verbatim}

\begin{Shaded}
\begin{Highlighting}[]
\CommentTok{\#ouverture des jeux de données de consommation d\textquotesingle{}électricité des ménages}
\NormalTok{celec}\OtherTok{\textless{}{-}}\FunctionTok{read\_excel}\NormalTok{(}\StringTok{"celec\_menages.xlsx"}\NormalTok{, }\AttributeTok{col\_names =} \ConstantTok{TRUE}\NormalTok{)}
\end{Highlighting}
\end{Shaded}

\begin{Shaded}
\begin{Highlighting}[]
\CommentTok{\#nous regardons la structure des données de la table "celec"}
\FunctionTok{str}\NormalTok{(celec)}
\end{Highlighting}
\end{Shaded}

\begin{verbatim}
## tibble [32 x 7] (S3: tbl_df/tbl/data.frame)
##  $ Date             : num [1:32] 1990 1991 1992 1993 1994 ...
##  $ IPC(base100=2015): num [1:32] 67.4 69.6 71.2 72.7 73.9 75.3 76.8 77.7 78.2 78.6 ...
##  $ PIB2020          : num [1:32] 1566 1586 1610 1604 1642 ...
##  $ Pelec            : num [1:32] 125 121 124 126 127 ...
##  $ Pop1             : num [1:32] 56708831 56975597 57239847 57467085 57658772 ...
##  $ DJU              : num [1:32] 0.96 1.167 1.076 1.076 0.922 ...
##  $ Celec_menages    : num [1:32] 96.9 106.8 109.6 111.5 111.2 ...
\end{verbatim}

IPC : indice annuel des prix à la consommation PIB2020 : produit
intérieur brut en euros de 2020 Pelec : Prix de l'électricité des
ménages (euro/MWh) Pop1 : population France métropolitaine DJU : Indice
de rigueur du climat (MTES) Celec\_menages : consommation électrice GWh
observée

\#\%=================================================\%= \#\# PREPARING
SET FOR EXPLORATION AND REGRESSION \#\#\#\#
\#\%=================================================\%=

\subsection{Renommer les colonnes pour une meilleure
clarté}\label{renommer-les-colonnes-pour-une-meilleure-clartuxe9}

\subsection{IRC = Indice de Rigueur
Climatique.}\label{irc-indice-de-rigueur-climatique.}

\begin{Shaded}
\begin{Highlighting}[]
\NormalTok{ celec }\OtherTok{\textless{}{-}}\NormalTok{ celec }\SpecialCharTok{\%\textgreater{}\%}
  \FunctionTok{rename}\NormalTok{(}
        \AttributeTok{IRC =}\NormalTok{ DJU,}
        \AttributeTok{Population =}\NormalTok{ Pop1,}
        \AttributeTok{elec\_cons =}\NormalTok{ Celec\_menages,}
        \AttributeTok{IPC =} \StringTok{\textasciigrave{}}\AttributeTok{IPC(base100=2015)}\StringTok{\textasciigrave{}}\NormalTok{,}
\NormalTok{    )}
\end{Highlighting}
\end{Shaded}

\subsection{Synthèse descriptive des
données}\label{synthuxe8se-descriptive-des-donnuxe9es}

\begin{Shaded}
\begin{Highlighting}[]
\FunctionTok{summary}\NormalTok{(celec)}
\end{Highlighting}
\end{Shaded}

\begin{verbatim}
##       Date           IPC            PIB2020         Pelec      
##  Min.   :1990   Min.   : 67.40   Min.   :1566   Min.   :109.9  
##  1st Qu.:1998   1st Qu.: 78.08   1st Qu.:1792   1st Qu.:114.1  
##  Median :2006   Median : 88.61   Median :2149   Median :124.5  
##  Mean   :2006   Mean   : 88.12   Mean   :2065   Mean   :133.6  
##  3rd Qu.:2013   3rd Qu.: 99.58   3rd Qu.:2303   3rd Qu.:148.4  
##  Max.   :2021   Max.   :106.45   Max.   :2505   Max.   :193.1  
##    Population            IRC           elec_cons     
##  Min.   :56708831   Min.   :0.8311   Min.   : 96.91  
##  1st Qu.:58350214   1st Qu.:0.9322   1st Qu.:122.49  
##  Median :61389492   Median :1.0034   Median :142.46  
##  Mean   :61214776   Mean   :0.9995   Mean   :138.43  
##  3rd Qu.:63938216   3rd Qu.:1.0475   3rd Qu.:156.87  
##  Max.   :65613522   Max.   :1.1943   Max.   :166.67
\end{verbatim}

\#\%=========================================== \#\# Going base 100 for
all variables in 2015

\begin{Shaded}
\begin{Highlighting}[]
\NormalTok{base\_2015 }\OtherTok{\textless{}{-}}\NormalTok{ celec }\SpecialCharTok{\%\textgreater{}\%} \FunctionTok{filter}\NormalTok{(Date }\SpecialCharTok{==} \DecValTok{2015}\NormalTok{)}
\end{Highlighting}
\end{Shaded}

\begin{Shaded}
\begin{Highlighting}[]
\NormalTok{celec }\OtherTok{\textless{}{-}}\NormalTok{ celec }\SpecialCharTok{\%\textgreater{}\%}
    \FunctionTok{mutate}\NormalTok{(}
        \AttributeTok{PIB2020\_base100\_2015 =}\NormalTok{ PIB2020 }\SpecialCharTok{/}\NormalTok{ base\_2015}\SpecialCharTok{$}\NormalTok{PIB2020 }\SpecialCharTok{*} \DecValTok{100}\NormalTok{,}
        \AttributeTok{Pelec\_base100\_2015 =}\NormalTok{ Pelec }\SpecialCharTok{/}\NormalTok{ base\_2015}\SpecialCharTok{$}\NormalTok{Pelec }\SpecialCharTok{*} \DecValTok{100}\NormalTok{,}
        \AttributeTok{Population\_base100\_2015 =}\NormalTok{ Population }\SpecialCharTok{/}\NormalTok{ base\_2015}\SpecialCharTok{$}\NormalTok{Population }\SpecialCharTok{*} \DecValTok{100}\NormalTok{,}
        \AttributeTok{IRC\_base100\_2015 =}\NormalTok{ IRC }\SpecialCharTok{/}\NormalTok{ base\_2015}\SpecialCharTok{$}\NormalTok{IRC }\SpecialCharTok{*} \DecValTok{100}\NormalTok{,}
        \AttributeTok{elec\_cons\_base100\_2015 =}\NormalTok{ elec\_cons }\SpecialCharTok{/}\NormalTok{ base\_2015}\SpecialCharTok{$}\NormalTok{elec\_cons }\SpecialCharTok{*} \DecValTok{100}\NormalTok{, }
        \AttributeTok{IPC\_base100\_2015 =}\NormalTok{ IPC  }\CommentTok{\# L\textquotesingle{}IPC est déjà en base 100}
\NormalTok{    )}
\end{Highlighting}
\end{Shaded}

\subsection{Tracer toutes les colonnes qui se terminent par
base100\_2015 en fonction de
Date}\label{tracer-toutes-les-colonnes-qui-se-terminent-par-base100_2015-en-fonction-de-date}

\subsection{Convertir les données en format long pour
ggplot}\label{convertir-les-donnuxe9es-en-format-long-pour-ggplot}

\begin{Shaded}
\begin{Highlighting}[]
\NormalTok{celec\_long }\OtherTok{\textless{}{-}}\NormalTok{ celec }\SpecialCharTok{\%\textgreater{}\%}
    \FunctionTok{select}\NormalTok{(Date, }\FunctionTok{ends\_with}\NormalTok{(}\StringTok{"base100\_2015"}\NormalTok{)) }\SpecialCharTok{\%\textgreater{}\%}
    \FunctionTok{pivot\_longer}\NormalTok{(}\AttributeTok{cols =} \SpecialCharTok{{-}}\NormalTok{Date, }\AttributeTok{names\_to =} \StringTok{"variable"}\NormalTok{, }\AttributeTok{values\_to =} \StringTok{"value"}\NormalTok{) }\SpecialCharTok{\%\textgreater{}\%}
    \FunctionTok{mutate}\NormalTok{(}\AttributeTok{variable =} \FunctionTok{gsub}\NormalTok{(}\StringTok{"\_base100\_2015"}\NormalTok{, }\StringTok{""}\NormalTok{, variable))}
\end{Highlighting}
\end{Shaded}

\subsection{Tracer les données}\label{tracer-les-donnuxe9es}

\begin{Shaded}
\begin{Highlighting}[]
\FunctionTok{ggplot}\NormalTok{(celec\_long, }\FunctionTok{aes}\NormalTok{(}\AttributeTok{x =}\NormalTok{ Date, }\AttributeTok{y =}\NormalTok{ value, }\AttributeTok{color =}\NormalTok{ variable)) }\SpecialCharTok{+}
    \FunctionTok{geom\_line}\NormalTok{(}\AttributeTok{linewidth =} \FloatTok{0.8}\NormalTok{) }\SpecialCharTok{+}
    \FunctionTok{geom\_point}\NormalTok{(}\AttributeTok{size =} \DecValTok{1}\NormalTok{) }\SpecialCharTok{+}
    \FunctionTok{scale\_color\_manual}\NormalTok{(}\AttributeTok{values =} \FunctionTok{c}\NormalTok{(}\StringTok{"red"}\NormalTok{, }\StringTok{"blue"}\NormalTok{, }\StringTok{"green"}\NormalTok{, }\StringTok{"purple"}\NormalTok{, }\StringTok{"orange"}\NormalTok{, }\StringTok{"\#00c3ff"}\NormalTok{)) }\SpecialCharTok{+}
    \FunctionTok{labs}\NormalTok{(}\AttributeTok{title =} \StringTok{"Évolution temporelle des variables (normalisées en base 100)"}\NormalTok{,}
       \AttributeTok{x =} \StringTok{"Date"}\NormalTok{,}
       \AttributeTok{y =} \StringTok{"Valeur (base 100 en 2015)"}\NormalTok{,}
       \AttributeTok{color =} \StringTok{"Variable"}\NormalTok{) }\SpecialCharTok{+}
    \FunctionTok{theme\_minimal}\NormalTok{()}
\end{Highlighting}
\end{Shaded}

\pandocbounded{\includegraphics[keepaspectratio]{EconometrieLea2_files/figure-latex/unnamed-chunk-9-1.pdf}}

\#\%=========================================== \# Adding new variables
for the regression

\section{Le PIB est en euro constant 2020 mais l'inflation en base 2015
: on doit ajuster le PIB en
2015}\label{le-pib-est-en-euro-constant-2020-mais-linflation-en-base-2015-on-doit-ajuster-le-pib-en-2015}

\begin{Shaded}
\begin{Highlighting}[]
\NormalTok{celec }\OtherTok{\textless{}{-}}\NormalTok{ celec }\SpecialCharTok{\%\textgreater{}\%}
    \FunctionTok{mutate}\NormalTok{(}
        \AttributeTok{elec\_cons\_pc =}\NormalTok{ elec\_cons }\SpecialCharTok{/}\NormalTok{ Population, }\CommentTok{\# pc = per capita}
        \AttributeTok{PIB2015 =}\NormalTok{ PIB2020 }\SpecialCharTok{*}\NormalTok{ (IPC[Date }\SpecialCharTok{==} \DecValTok{2015}\NormalTok{] }\SpecialCharTok{/}\NormalTok{ IPC[Date }\SpecialCharTok{==} \DecValTok{2020}\NormalTok{]), }\CommentTok{\#en milliards d\textquotesingle{}euros 2015}
        \AttributeTok{PIB2015\_pc =}\NormalTok{ PIB2015 }\SpecialCharTok{/}\NormalTok{ Population, }\CommentTok{\#in 2015 10\^{}9 euros per capita, }
        \AttributeTok{Pelec\_euro2015 =}\NormalTok{ Pelec }\SpecialCharTok{*}\NormalTok{ (IPC[Date }\SpecialCharTok{==} \DecValTok{2015}\NormalTok{] }\SpecialCharTok{/}\NormalTok{ IPC), }\CommentTok{\# Prix de l\textquotesingle{}électricité en euro constant 2015}
\NormalTok{    )}
\end{Highlighting}
\end{Shaded}

\#\%==========================================================================\%=
\#\# PREMIERE REGRESSION GLOBALE SANS PRISE EN COMPTE DE LA RUPTURE EN
2009 \#\#\#\#
\#\%==========================================================================\%=

\section{Régression}\label{ruxe9gression}

\begin{Shaded}
\begin{Highlighting}[]
\NormalTok{celec.lm}\OtherTok{=}\FunctionTok{lm}\NormalTok{(elec\_cons}\SpecialCharTok{\textasciitilde{}}\NormalTok{PIB2015 }\SpecialCharTok{+}\NormalTok{ Population }\SpecialCharTok{+}\NormalTok{ IRC }\SpecialCharTok{+}\NormalTok{ Pelec\_euro2015 }\SpecialCharTok{+}\NormalTok{ IPC }\SpecialCharTok{+}\NormalTok{ Date, }\AttributeTok{data =}\NormalTok{ celec)}
\FunctionTok{par}\NormalTok{(}\AttributeTok{mfrow=}\FunctionTok{c}\NormalTok{(}\DecValTok{2}\NormalTok{,}\DecValTok{2}\NormalTok{))}
\FunctionTok{plot}\NormalTok{(celec.lm)}
\end{Highlighting}
\end{Shaded}

\pandocbounded{\includegraphics[keepaspectratio]{EconometrieLea2_files/figure-latex/unnamed-chunk-11-1.pdf}}

\begin{Shaded}
\begin{Highlighting}[]
\FunctionTok{summary}\NormalTok{(celec.lm)}
\end{Highlighting}
\end{Shaded}

\begin{verbatim}
## 
## Call:
## lm(formula = elec_cons ~ PIB2015 + Population + IRC + Pelec_euro2015 + 
##     IPC + Date, data = celec)
## 
## Residuals:
##     Min      1Q  Median      3Q     Max 
## -5.8659 -1.3512  0.4899  1.6491  5.1223 
## 
## Coefficients:
##                  Estimate Std. Error t value Pr(>|t|)    
## (Intercept)    -4.905e+03  1.531e+03  -3.205  0.00367 ** 
## PIB2015        -1.389e-02  1.479e-02  -0.939  0.35671    
## Population     -3.636e-07  2.591e-06  -0.140  0.88953    
## IRC             4.093e+01  6.044e+00   6.771 4.26e-07 ***
## Pelec_euro2015 -2.348e-01  4.814e-02  -4.876 5.13e-05 ***
## IPC             8.673e-02  7.253e-01   0.120  0.90577    
## Date            2.533e+00  7.893e-01   3.210  0.00363 ** 
## ---
## Signif. codes:  0 '***' 0.001 '**' 0.01 '*' 0.05 '.' 0.1 ' ' 1
## 
## Residual standard error: 2.742 on 25 degrees of freedom
## Multiple R-squared:  0.9847, Adjusted R-squared:  0.981 
## F-statistic: 267.5 on 6 and 25 DF,  p-value: < 2.2e-16
\end{verbatim}

Les variables PIB2015, Population et IPC ont des p-value supérieures à
5\% donc nous les retirons.

\section{Deuxième régression}\label{deuxiuxe8me-ruxe9gression}

\begin{Shaded}
\begin{Highlighting}[]
\NormalTok{celec.lm2}\OtherTok{=}\FunctionTok{lm}\NormalTok{(elec\_cons}\SpecialCharTok{\textasciitilde{}}\NormalTok{IRC }\SpecialCharTok{+}\NormalTok{ Pelec\_euro2015 }\SpecialCharTok{+}\NormalTok{ Date, }\AttributeTok{data =}\NormalTok{ celec)}
\FunctionTok{par}\NormalTok{(}\AttributeTok{mfrow=}\FunctionTok{c}\NormalTok{(}\DecValTok{2}\NormalTok{,}\DecValTok{2}\NormalTok{))}
\FunctionTok{plot}\NormalTok{(celec.lm2)}
\end{Highlighting}
\end{Shaded}

\pandocbounded{\includegraphics[keepaspectratio]{EconometrieLea2_files/figure-latex/unnamed-chunk-12-1.pdf}}

\begin{Shaded}
\begin{Highlighting}[]
\FunctionTok{summary}\NormalTok{(celec.lm2)}
\end{Highlighting}
\end{Shaded}

\begin{verbatim}
## 
## Call:
## lm(formula = elec_cons ~ IRC + Pelec_euro2015 + Date, data = celec)
## 
## Residuals:
##     Min      1Q  Median      3Q     Max 
## -6.1082 -1.5322  0.5801  1.6865  5.2570 
## 
## Coefficients:
##                  Estimate Std. Error t value Pr(>|t|)    
## (Intercept)    -4.124e+03  1.116e+02 -36.967  < 2e-16 ***
## IRC             4.122e+01  5.700e+00   7.231 7.16e-08 ***
## Pelec_euro2015 -2.021e-01  2.447e-02  -8.258 5.49e-09 ***
## Date            2.120e+00  5.434e-02  39.019  < 2e-16 ***
## ---
## Signif. codes:  0 '***' 0.001 '**' 0.01 '*' 0.05 '.' 0.1 ' ' 1
## 
## Residual standard error: 2.654 on 28 degrees of freedom
## Multiple R-squared:  0.9839, Adjusted R-squared:  0.9822 
## F-statistic: 570.5 on 3 and 28 DF,  p-value: < 2.2e-16
\end{verbatim}

\section{Vérification de la normalité des
résidus}\label{vuxe9rification-de-la-normalituxe9-des-ruxe9sidus}

\subsection{Histogramme des résidus}\label{histogramme-des-ruxe9sidus}

\begin{Shaded}
\begin{Highlighting}[]
\FunctionTok{hist}\NormalTok{(}\FunctionTok{residuals}\NormalTok{(celec.lm2), }\AttributeTok{main=}\StringTok{"Histogramme des résidus"}\NormalTok{, }\AttributeTok{xlab=}\StringTok{"Résidus"}\NormalTok{)}
\end{Highlighting}
\end{Shaded}

\pandocbounded{\includegraphics[keepaspectratio]{EconometrieLea2_files/figure-latex/unnamed-chunk-13-1.pdf}}
\#\# Test de Shapiro-Wilk

\begin{Shaded}
\begin{Highlighting}[]
\FunctionTok{shapiro.test}\NormalTok{(}\FunctionTok{residuals}\NormalTok{(celec.lm2))}
\end{Highlighting}
\end{Shaded}

\begin{verbatim}
## 
##  Shapiro-Wilk normality test
## 
## data:  residuals(celec.lm2)
## W = 0.9807, p-value = 0.8196
\end{verbatim}

Normalité validée

\subsection{Test d'Anderson-Darling}\label{test-danderson-darling}

\begin{Shaded}
\begin{Highlighting}[]
\FunctionTok{ad.test}\NormalTok{(}\FunctionTok{residuals}\NormalTok{(celec.lm2))}
\end{Highlighting}
\end{Shaded}

\begin{verbatim}
## 
##  Anderson-Darling normality test
## 
## data:  residuals(celec.lm2)
## A = 0.31579, p-value = 0.5251
\end{verbatim}

Normalité validée

\section{Vérification de
l'hétéroscédasticité}\label{vuxe9rification-de-lhuxe9tuxe9roscuxe9dasticituxe9}

\subsection{Test de Breusch-Pagan}\label{test-de-breusch-pagan}

\begin{Shaded}
\begin{Highlighting}[]
\FunctionTok{bptest}\NormalTok{(celec.lm2)}
\end{Highlighting}
\end{Shaded}

\begin{verbatim}
## 
##  studentized Breusch-Pagan test
## 
## data:  celec.lm2
## BP = 1.8721, df = 3, p-value = 0.5994
\end{verbatim}

Homoscédasticité validée

\subsection{Test de White}\label{test-de-white}

\begin{Shaded}
\begin{Highlighting}[]
\FunctionTok{bptest}\NormalTok{(celec.lm2, }\SpecialCharTok{\textasciitilde{}} \FunctionTok{fitted}\NormalTok{(celec.lm2) }\SpecialCharTok{+} \FunctionTok{I}\NormalTok{(}\FunctionTok{fitted}\NormalTok{(celec.lm2)))}
\end{Highlighting}
\end{Shaded}

\begin{verbatim}
## 
##  studentized Breusch-Pagan test
## 
## data:  celec.lm2
## BP = 0.12413, df = 1, p-value = 0.7246
\end{verbatim}

\section{Test d'autocorrélation}\label{test-dautocorruxe9lation}

\subsection{Test de Breusch-Godfrey}\label{test-de-breusch-godfrey}

\begin{Shaded}
\begin{Highlighting}[]
\FunctionTok{bgtest}\NormalTok{(celec.lm2)}
\end{Highlighting}
\end{Shaded}

\begin{verbatim}
## 
##  Breusch-Godfrey test for serial correlation of order up to 1
## 
## data:  celec.lm2
## LM test = 0.038089, df = 1, p-value = 0.8453
\end{verbatim}

Pas d'autocorrélation.

\section{Etude de la
multicolinéarité}\label{etude-de-la-multicolinuxe9arituxe9}

\begin{Shaded}
\begin{Highlighting}[]
\FunctionTok{vif}\NormalTok{(celec.lm2)}
\end{Highlighting}
\end{Shaded}

\begin{verbatim}
##            IRC Pelec_euro2015           Date 
##       1.147436       1.020445       1.143243
\end{verbatim}

VIF pas élevé (inférieur à 5) donc pas d'autocorrelation notable.

\begin{Shaded}
\begin{Highlighting}[]
\FunctionTok{vif}\NormalTok{(celec.lm)}
\end{Highlighting}
\end{Shaded}

\begin{verbatim}
##        PIB2015     Population            IRC Pelec_euro2015            IPC 
##      73.898719     239.556870       1.209056       3.701178     304.932156 
##           Date 
##     226.017040
\end{verbatim}

Dans le premier modèle, les VIF étaient très élevées.

-\textgreater{} Donc pas besoin de passer par du Lasso et de l'ACP.

\section{Diagnostic visuel}\label{diagnostic-visuel}

\subsection{Graphique des valeurs ajustées vs
résidus}\label{graphique-des-valeurs-ajustuxe9es-vs-ruxe9sidus}

\begin{Shaded}
\begin{Highlighting}[]
\FunctionTok{plot}\NormalTok{(}\FunctionTok{fitted}\NormalTok{(celec.lm2), }\FunctionTok{residuals}\NormalTok{(celec.lm2), }\AttributeTok{main=}\StringTok{"Valeurs ajustées vs Résidus"}\NormalTok{, }\AttributeTok{xlab=}\StringTok{"Valeurs ajustées"}\NormalTok{, }\AttributeTok{ylab=}\StringTok{"Résidus"}\NormalTok{)}
\FunctionTok{abline}\NormalTok{(}\AttributeTok{h=}\DecValTok{0}\NormalTok{, }\AttributeTok{col=}\StringTok{"red"}\NormalTok{)}
\end{Highlighting}
\end{Shaded}

\pandocbounded{\includegraphics[keepaspectratio]{EconometrieLea2_files/figure-latex/unnamed-chunk-21-1.pdf}}

Perfetto

\subsection{QQ-plot des résidus}\label{qq-plot-des-ruxe9sidus}

\begin{Shaded}
\begin{Highlighting}[]
\FunctionTok{qqnorm}\NormalTok{(}\FunctionTok{residuals}\NormalTok{(celec.lm2))}
\FunctionTok{qqline}\NormalTok{(}\FunctionTok{residuals}\NormalTok{(celec.lm2), }\AttributeTok{col=}\StringTok{"red"}\NormalTok{)}
\end{Highlighting}
\end{Shaded}

\pandocbounded{\includegraphics[keepaspectratio]{EconometrieLea2_files/figure-latex/unnamed-chunk-22-1.pdf}}
ça serpente un peu quand même.

\#\%==============================================================\%=
\#\# DEUXIEME MODELE AVEC PRISE EN COMPTE DE LA RUPTURE EN 2009 \#\#\#\#
\#\%==============================================================\%=

\begin{Shaded}
\begin{Highlighting}[]
\NormalTok{celec.lm3}\OtherTok{=}\FunctionTok{lm}\NormalTok{(}\FunctionTok{log}\NormalTok{(elec\_cons\_pc)}\SpecialCharTok{\textasciitilde{}}\FunctionTok{log}\NormalTok{(PIB2015\_pc) }\SpecialCharTok{+} \FunctionTok{log}\NormalTok{(Pelec\_euro2015) }\SpecialCharTok{+}\NormalTok{ IRC , }\AttributeTok{data =}\NormalTok{ celec)}
\FunctionTok{par}\NormalTok{(}\AttributeTok{mfrow=}\FunctionTok{c}\NormalTok{(}\DecValTok{2}\NormalTok{,}\DecValTok{2}\NormalTok{))}
\FunctionTok{plot}\NormalTok{(celec.lm3)}
\end{Highlighting}
\end{Shaded}

\pandocbounded{\includegraphics[keepaspectratio]{EconometrieLea2_files/figure-latex/unnamed-chunk-23-1.pdf}}

\begin{Shaded}
\begin{Highlighting}[]
\FunctionTok{summary}\NormalTok{(celec.lm3)}
\end{Highlighting}
\end{Shaded}

\begin{verbatim}
## 
## Call:
## lm(formula = log(elec_cons_pc) ~ log(PIB2015_pc) + log(Pelec_euro2015) + 
##     IRC, data = celec)
## 
## Residuals:
##       Min        1Q    Median        3Q       Max 
## -0.061256 -0.015346  0.004301  0.018180  0.071845 
## 
## Coefficients:
##                     Estimate Std. Error t value Pr(>|t|)    
## (Intercept)         -2.70320    0.54414  -4.968 3.03e-05 ***
## log(PIB2015_pc)      1.00773    0.05931  16.992 2.78e-16 ***
## log(Pelec_euro2015) -0.03134    0.04537  -0.691 0.495441    
## IRC                  0.28438    0.06487   4.384 0.000149 ***
## ---
## Signif. codes:  0 '***' 0.001 '**' 0.01 '*' 0.05 '.' 0.1 ' ' 1
## 
## Residual standard error: 0.02995 on 28 degrees of freedom
## Multiple R-squared:  0.9279, Adjusted R-squared:  0.9202 
## F-statistic: 120.2 on 3 and 28 DF,  p-value: 4.273e-16
\end{verbatim}

The coefficients β1 and β2 are, respectively, the income elasticity of
electricity consumption and the price elasticity of electricity
consumption. These coefficients measure the percentage change in
electricity consumption in relation, respectively, to a percentage
change in income and a percentage change in prices. Here β1 is estimated
1.01 and β2 is estimated -0.03. Chow test could be not applicable if
residuals are not independant.

test de Chow pas immédiatement applicable ? -\textgreater{} les résidus
du modèle doivent être indépendants et ne pas montrer de tendance
-\textgreater{} or les graphs sont un peu dégueux ?

\begin{Shaded}
\begin{Highlighting}[]
\CommentTok{\# Création de deux groupes pour application du test de Chow}
\NormalTok{groupe1 }\OtherTok{\textless{}{-}} \FunctionTok{subset}\NormalTok{(celec, Date}\SpecialCharTok{\textless{}}\DecValTok{2009}\NormalTok{)}
\NormalTok{groupe2 }\OtherTok{\textless{}{-}} \FunctionTok{subset}\NormalTok{(celec, Date}\SpecialCharTok{\textgreater{}=}\DecValTok{2009}\NormalTok{)}
\end{Highlighting}
\end{Shaded}

\begin{Shaded}
\begin{Highlighting}[]
\NormalTok{lm1\_chow }\OtherTok{\textless{}{-}} \FunctionTok{lm}\NormalTok{(}\FunctionTok{log}\NormalTok{(elec\_cons\_pc) }\SpecialCharTok{\textasciitilde{}} \FunctionTok{log}\NormalTok{(PIB2015\_pc) }\SpecialCharTok{+} \FunctionTok{log}\NormalTok{(Pelec\_euro2015) }\SpecialCharTok{+}\NormalTok{ IRC, }\AttributeTok{data =}\NormalTok{ groupe1)}
\NormalTok{lm2\_chow }\OtherTok{\textless{}{-}} \FunctionTok{lm}\NormalTok{(}\FunctionTok{log}\NormalTok{(elec\_cons\_pc) }\SpecialCharTok{\textasciitilde{}} \FunctionTok{log}\NormalTok{(PIB2015\_pc) }\SpecialCharTok{+} \FunctionTok{log}\NormalTok{(Pelec\_euro2015) }\SpecialCharTok{+}\NormalTok{ IRC, }\AttributeTok{data =}\NormalTok{ groupe2)}
\end{Highlighting}
\end{Shaded}

\begin{Shaded}
\begin{Highlighting}[]
\CommentTok{\#Ajout de l\textquotesingle{}indication groupe 1 et groupe 2 dans la table celec}
\NormalTok{celec}\SpecialCharTok{$}\NormalTok{group }\OtherTok{\textless{}{-}} \FunctionTok{ifelse}\NormalTok{(celec}\SpecialCharTok{$}\NormalTok{Date }\SpecialCharTok{\textless{}} \DecValTok{2009}\NormalTok{, }\StringTok{"groupe1"}\NormalTok{, }\StringTok{"groupe2"}\NormalTok{)}
\end{Highlighting}
\end{Shaded}

\begin{Shaded}
\begin{Highlighting}[]
\NormalTok{lm\_global }\OtherTok{\textless{}{-}} \FunctionTok{lm}\NormalTok{(}\FunctionTok{log}\NormalTok{(elec\_cons\_pc) }\SpecialCharTok{\textasciitilde{}} \FunctionTok{log}\NormalTok{(PIB2015\_pc) }\SpecialCharTok{*}\NormalTok{ group }\SpecialCharTok{+}
                                  \FunctionTok{log}\NormalTok{(Pelec\_euro2015) }\SpecialCharTok{*}\NormalTok{ group }\SpecialCharTok{+}
\NormalTok{                                  IRC }\SpecialCharTok{*}\NormalTok{ group,}
                \AttributeTok{data =}\NormalTok{ celec)}
\end{Highlighting}
\end{Shaded}

\begin{Shaded}
\begin{Highlighting}[]
\CommentTok{\# Test de Chow}
\FunctionTok{linearHypothesis}\NormalTok{(lm\_global, }
                 \FunctionTok{c}\NormalTok{(}\StringTok{"log(PIB2015\_pc):groupgroupe2 = 0"}\NormalTok{,}
                   \StringTok{"groupgroupe2:log(Pelec\_euro2015) = 0"}\NormalTok{,}
                   \StringTok{"groupgroupe2:IRC = 0"}\NormalTok{))}
\end{Highlighting}
\end{Shaded}

\begin{verbatim}
## Linear hypothesis test
## 
## Hypothesis:
## log(PIB2015_pc):groupgroupe2 = 0
## groupgroupe2:log(Pelec_euro2015) = 0
## groupgroupe2:IRC = 0
## 
## Model 1: restricted model
## Model 2: log(elec_cons_pc) ~ log(PIB2015_pc) * group + log(Pelec_euro2015) * 
##     group + IRC * group
## 
##   Res.Df       RSS Df Sum of Sq      F    Pr(>F)    
## 1     27 0.0229376                                  
## 2     24 0.0093388  3  0.013599 11.649 6.603e-05 ***
## ---
## Signif. codes:  0 '***' 0.001 '**' 0.01 '*' 0.05 '.' 0.1 ' ' 1
\end{verbatim}

The Chow test returns a p-value lower than 0.05, so we can validate a
break in 2009. This implies that we need to different models to fit with
the data before and after 2009. Thus, our projection for 2030 have to be
estimated using the model fitting data after 2009.

\begin{Shaded}
\begin{Highlighting}[]
\CommentTok{\#Creation de la table qui ne contient que les données supérieures à 2009}
\NormalTok{celec\_2009 }\OtherTok{\textless{}{-}}\NormalTok{ celec[celec}\SpecialCharTok{$}\NormalTok{Date }\SpecialCharTok{\textgreater{}=} \DecValTok{2009}\NormalTok{, ]}
\end{Highlighting}
\end{Shaded}

\section{Modele de régression à partir de
2009}\label{modele-de-ruxe9gression-uxe0-partir-de-2009}

\begin{Shaded}
\begin{Highlighting}[]
\NormalTok{celec.lm4}\OtherTok{=}\FunctionTok{lm}\NormalTok{(}\FunctionTok{log}\NormalTok{(elec\_cons\_pc)}\SpecialCharTok{\textasciitilde{}}\FunctionTok{log}\NormalTok{(PIB2015\_pc) }\SpecialCharTok{+} \FunctionTok{log}\NormalTok{(Pelec\_euro2015) }\SpecialCharTok{+}\NormalTok{ IRC, }\AttributeTok{data =}\NormalTok{ celec\_2009)}
\FunctionTok{par}\NormalTok{(}\AttributeTok{mfrow=}\FunctionTok{c}\NormalTok{(}\DecValTok{2}\NormalTok{,}\DecValTok{2}\NormalTok{))}
\FunctionTok{plot}\NormalTok{(celec.lm4)}
\end{Highlighting}
\end{Shaded}

\pandocbounded{\includegraphics[keepaspectratio]{EconometrieLea2_files/figure-latex/unnamed-chunk-30-1.pdf}}

\begin{Shaded}
\begin{Highlighting}[]
\FunctionTok{summary}\NormalTok{(celec.lm4)}
\end{Highlighting}
\end{Shaded}

\begin{verbatim}
## 
## Call:
## lm(formula = log(elec_cons_pc) ~ log(PIB2015_pc) + log(Pelec_euro2015) + 
##     IRC, data = celec_2009)
## 
## Residuals:
##       Min        1Q    Median        3Q       Max 
## -0.013633 -0.004151 -0.002211  0.004489  0.013742 
## 
## Coefficients:
##                      Estimate Std. Error t value Pr(>|t|)    
## (Intercept)         -17.55915    1.48111 -11.855 8.54e-07 ***
## log(PIB2015_pc)      -0.31061    0.13127  -2.366 0.042169 *  
## log(Pelec_euro2015)   0.21582    0.03476   6.208 0.000157 ***
## IRC                   0.36781    0.02792  13.172 3.47e-07 ***
## ---
## Signif. codes:  0 '***' 0.001 '**' 0.01 '*' 0.05 '.' 0.1 ' ' 1
## 
## Residual standard error: 0.00939 on 9 degrees of freedom
## Multiple R-squared:  0.9509, Adjusted R-squared:  0.9346 
## F-statistic: 58.12 on 3 and 9 DF,  p-value: 3.26e-06
\end{verbatim}

\begin{Shaded}
\begin{Highlighting}[]
\FunctionTok{step}\NormalTok{(celec.lm4)}
\end{Highlighting}
\end{Shaded}

\begin{verbatim}
## Start:  AIC=-118.15
## log(elec_cons_pc) ~ log(PIB2015_pc) + log(Pelec_euro2015) + IRC
## 
##                       Df Sum of Sq       RSS      AIC
## <none>                             0.0007936 -118.150
## - log(PIB2015_pc)      1 0.0004937 0.0012873 -113.862
## - log(Pelec_euro2015)  1 0.0033985 0.0041921  -98.513
## - IRC                  1 0.0152999 0.0160935  -81.026
\end{verbatim}

\begin{verbatim}
## 
## Call:
## lm(formula = log(elec_cons_pc) ~ log(PIB2015_pc) + log(Pelec_euro2015) + 
##     IRC, data = celec_2009)
## 
## Coefficients:
##         (Intercept)      log(PIB2015_pc)  log(Pelec_euro2015)  
##            -17.5592              -0.3106               0.2158  
##                 IRC  
##              0.3678
\end{verbatim}

The Akaike criteria tells us that there's no variable that we can remove
from the model without loosing too much information.

\begin{Shaded}
\begin{Highlighting}[]
\CommentTok{\#un autre modele, mais que je n\textquotesingle{}exploite pas par la suite}
\NormalTok{celec.lm5}\OtherTok{=}\FunctionTok{lm}\NormalTok{(elec\_cons\_pc}\SpecialCharTok{\textasciitilde{}}\NormalTok{ IRC }\SpecialCharTok{+}\NormalTok{ PIB2015\_pc }\SpecialCharTok{+}\NormalTok{ Population, }\AttributeTok{data =}\NormalTok{ celec\_2009)}
\FunctionTok{par}\NormalTok{(}\AttributeTok{mfrow=}\FunctionTok{c}\NormalTok{(}\DecValTok{2}\NormalTok{,}\DecValTok{2}\NormalTok{))}
\FunctionTok{plot}\NormalTok{(celec.lm5)}
\end{Highlighting}
\end{Shaded}

\pandocbounded{\includegraphics[keepaspectratio]{EconometrieLea2_files/figure-latex/unnamed-chunk-32-1.pdf}}

\begin{Shaded}
\begin{Highlighting}[]
\FunctionTok{summary}\NormalTok{(celec.lm5)}
\end{Highlighting}
\end{Shaded}

\begin{verbatim}
## 
## Call:
## lm(formula = elec_cons_pc ~ IRC + PIB2015_pc + Population, data = celec_2009)
## 
## Residuals:
##        Min         1Q     Median         3Q        Max 
## -3.819e-08 -9.283e-09 -5.038e-09  9.407e-09  3.534e-08 
## 
## Coefficients:
##               Estimate Std. Error t value Pr(>|t|)    
## (Intercept) -1.922e-06  5.582e-07  -3.443 0.007355 ** 
## IRC          8.854e-07  6.803e-08  13.015 3.85e-07 ***
## PIB2015_pc  -2.662e-02  9.784e-03  -2.720 0.023590 *  
## Population   6.890e-14  1.128e-14   6.109 0.000177 ***
## ---
## Signif. codes:  0 '***' 0.001 '**' 0.01 '*' 0.05 '.' 0.1 ' ' 1
## 
## Residual standard error: 2.317e-08 on 9 degrees of freedom
## Multiple R-squared:  0.9499, Adjusted R-squared:  0.9332 
## F-statistic: 56.92 on 3 and 9 DF,  p-value: 3.561e-06
\end{verbatim}

\section{Vérification de la normalité des
résidus}\label{vuxe9rification-de-la-normalituxe9-des-ruxe9sidus-1}

\subsection{Histogramme des résidus}\label{histogramme-des-ruxe9sidus-1}

\begin{Shaded}
\begin{Highlighting}[]
\FunctionTok{hist}\NormalTok{(}\FunctionTok{residuals}\NormalTok{(celec.lm4), }\AttributeTok{main=}\StringTok{"Histogramme des résidus"}\NormalTok{, }\AttributeTok{xlab=}\StringTok{"Résidus"}\NormalTok{)}
\end{Highlighting}
\end{Shaded}

\pandocbounded{\includegraphics[keepaspectratio]{EconometrieLea2_files/figure-latex/unnamed-chunk-33-1.pdf}}
Visually, there's a hint of non-normality, as a further increase can be
seen on the right-hand side of the graph. However, the data still has a
Gaussian shape, so we need to examine the normality of the residuals
further. The Shapiro--Wilk test is known not to work well in samples
with many identical values and Jarque-Bera is bad for small samples as
ours.The best test we can use seems to be Anderson-Darling.

\#\#Test d'Anderson-Darling

\begin{Shaded}
\begin{Highlighting}[]
\FunctionTok{ad.test}\NormalTok{(}\FunctionTok{residuals}\NormalTok{(celec.lm4))}
\end{Highlighting}
\end{Shaded}

\begin{verbatim}
## 
##  Anderson-Darling normality test
## 
## data:  residuals(celec.lm4)
## A = 0.26891, p-value = 0.6187
\end{verbatim}

The Anderson-Darling's test returns a p-value greater than 0.05 so we
can consider that the residuals follows a gaussian distribution.

\section{Vérification de
l'hétéroscédasticité}\label{vuxe9rification-de-lhuxe9tuxe9roscuxe9dasticituxe9-1}

To test for heteroscedasticity, we can choose between several tests.
Since the Goldfeld-Quandt test is not very robust to specification
errors and the White test is certainly more general and can detect a
wider range of forms of heteroscedasticity, but cannot be used for small
samples, we decide to use the Breusch-Pagan test. The latter is designed
to detect only linear forms of heteroscedasticity, which could be our
case.

\#\#Test de Breusch-Pagan

\begin{Shaded}
\begin{Highlighting}[]
\FunctionTok{bptest}\NormalTok{(celec.lm4)}
\end{Highlighting}
\end{Shaded}

\begin{verbatim}
## 
##  studentized Breusch-Pagan test
## 
## data:  celec.lm4
## BP = 2.8577, df = 3, p-value = 0.4141
\end{verbatim}

Here the test returns a p-value greater than 0.05 so we don't reject the
null hypothesis and we assume homoscedasticity.

\section{Test d'autocorrélation}\label{test-dautocorruxe9lation-1}

\#\#Test de Breusch-Godfrey

\begin{Shaded}
\begin{Highlighting}[]
\FunctionTok{bgtest}\NormalTok{(celec.lm4)}
\end{Highlighting}
\end{Shaded}

\begin{verbatim}
## 
##  Breusch-Godfrey test for serial correlation of order up to 1
## 
## data:  celec.lm4
## LM test = 1.6596, df = 1, p-value = 0.1977
\end{verbatim}

\#\#Etude de la multicolinéarité

\begin{Shaded}
\begin{Highlighting}[]
\FunctionTok{vif}\NormalTok{(celec.lm4)}
\end{Highlighting}
\end{Shaded}

\begin{verbatim}
##     log(PIB2015_pc) log(Pelec_euro2015)                 IRC 
##            2.046146            2.408974            1.259673
\end{verbatim}

Pas de multicolinéarité

\#Diagnostic visuel

\#\#Graphique des valeurs ajustées vs résidus

\begin{Shaded}
\begin{Highlighting}[]
\FunctionTok{plot}\NormalTok{(}\FunctionTok{fitted}\NormalTok{(celec.lm4), }\FunctionTok{residuals}\NormalTok{(celec.lm4), }\AttributeTok{main=}\StringTok{"Valeurs ajustées vs Résidus"}\NormalTok{, }\AttributeTok{xlab=}\StringTok{"Valeurs ajustées"}\NormalTok{, }\AttributeTok{ylab=}\StringTok{"Résidus"}\NormalTok{)}
\FunctionTok{abline}\NormalTok{(}\AttributeTok{h=}\DecValTok{0}\NormalTok{, }\AttributeTok{col=}\StringTok{"red"}\NormalTok{)}
\end{Highlighting}
\end{Shaded}

\pandocbounded{\includegraphics[keepaspectratio]{EconometrieLea2_files/figure-latex/unnamed-chunk-38-1.pdf}}

\#\#QQ-plot des résidus

\begin{Shaded}
\begin{Highlighting}[]
\FunctionTok{qqnorm}\NormalTok{(}\FunctionTok{residuals}\NormalTok{(celec.lm4))}
\FunctionTok{qqline}\NormalTok{(}\FunctionTok{residuals}\NormalTok{(celec.lm4), }\AttributeTok{col=}\StringTok{"red"}\NormalTok{)}
\end{Highlighting}
\end{Shaded}

\pandocbounded{\includegraphics[keepaspectratio]{EconometrieLea2_files/figure-latex/unnamed-chunk-39-1.pdf}}

\end{document}
