\section{Notes en vrac}
\begin{itemize}
    \item Synthèse: 5 à 10 pages. Might be something good to write in english.
    \item Goal: electricity demand modeling, with forecasting for 2030. 
    \item Observer la \textbf{relation entre prix de l'électricité et changement dans la structure des moyens de production} ces dernières années $\to$ besoins de meilleures variables ?? Ou juste regarder le PIB, contrôler pour l'IPC, etc.
    \item \textbf{impact des marchés de l'électricité sur la demande d'électricité} $\to$ on juge que le prix de l'électricité capture l'effet du "marché" ??
\end{itemize}

On n'a \textit{a priori} pas d'information sur la relation analytique entre les variables : on ne suppose aucune contrainte sur les estimateurs.



\subsection{Description}
\begin{enumerate}
    \item Relation économétrique entre les variables. 
    \item Prévision de la variable dépendante.
\end{enumerate}


\subsection{Notes cours C. Doz}

\subsubsection{Univariate time series}
Stationary around a deterministic trend: $X_t = a + bt + Y_t$ where $(Y_t)_t \in Z$ is a stationary process. (Stationarity if esperance and variance does not depend on $t$).

White noise: variance is constant, no autocorrelation, mean is zero.

Wold theorem: any stationary process can be written as a linear combination of white noise. $X_t = m + \sum_{i=0}^{\infty} \psi_i \varepsilon_{t-i}$. \\


Lag operator: $LX_t = X_{t-1}$. Now, if $(X_t)_t$ is a stationary process:
\begin{itemize}
    \item $AR(p)$ process: $X_t = \mu + \sum_{i=1}^{p} \phi_i X_{t-i} + \varepsilon_t$.
    \item Best Linear Forecast of an $AR(p)$ process: $X^*_{t+1\vert t} = \mu + \sum_{i=1}^{p} \phi_i X_{t+1-i} + \varepsilon_t$.
    \item Moving average process $MA(q)$: $X_t = \mu + \varepsilon_t + \sum_{i=1}^{q} \theta_i \varepsilon_{t-i}$.
    \item $ARMA(p, q)$ process: $X_t = \mu + \sum_{i=1}^{p} \phi_i X_{t-i} + \varepsilon_t + \sum_{j=1}^{q} \theta_j \varepsilon_{t-j}$.
    \item \textbf{If $(X_t)_t$ is an $ARMA(p, q)$ process, autocorrelation should tend exponentially to 0 with increasing lags.}
    \item On these processes, the impact of a shock is transitory.
\end{itemize}

\textit{In our case, electricity consumption might not be ARMA processes, so we might not be able to forecast as such... it is to be tested. We have to test which variables can be considered stationary around a deterministic trend.}

Now, if $X_t = \mu + X_{t-1} + \varepsilon_t$ (random walk), we have: 
\begin{itemize}
    \item ARIMA: $(1 - L)^d \Phi(L)X_t = \mu + \theta(L) \varepsilon$
    \item On ARIMA process, the impact of a shock is permanent.
    \item Autocorrelation of $X_t$ don't exponentially tend to 0 with increasing lags.
    \item Identication and estimation of an $ARIMA(p,d,q)$ :
    \begin{enumerate}
    \item choice of d : visual inspection of the estimated
    autocorrelogram + unit root tests (see below).
    \item If $(X_t)_t$ appears to be non-stationary, study $(1-L)X_t$, etc...
    \item Choose the smallest $d$ such that $(1-L)^d X_t$ appears to be stationary.
    \item choice of (p,q) : compute $Y_t = (1-L)^d X_t$ and apply to $Y_t$ the procedure which has been presented for ARMA(p,q).
    Estimate an ARMA model for $Y_t$.
    \end{enumerate}
\end{itemize}

\subsubsection{Multivariate time series}
Let's consider a vector of time series $(X_t)_t$ with $X_t = (X_{1t}, X_{2t}, \ldots, X_{kt})$. We suppose that $(X_t)_t$ is a stationary process. 

Wold theorem: 
If $(X_t)_t$ is a stationary process and $(\varepsilon_t)_t$ is a white noise, then $(X_t)_t$ can be written as a linear combination of $(\varepsilon_t)_t$: $$X_t = m + \sum_{i=0}^{\infty} A_i \varepsilon_{t-i}, \quad A_0 = I, \sum_{i = 1}^\infty A_i < \infty$$ \\

VAR(p): $X_t = \mu + \sum_{i=1}^{p} \Phi_i X_{t-i} + \varepsilon_t \Leftrightarrow  \Phi(L)X_t = \mu + \varepsilon_t$. \\

\textit{It's not clear that IRC is a good variable to include in the VAR model: it does not seem to be an AR(1) process, or at least not with our granularity.}

To do an IRF: Cholesky decomposition to have an orthogonalized impulse response function.

\subsection{Notes Ferrara - Doz}
\begin{enumerate}
    \item Data analysis
    \item Model specification
    \item Parameter estimation
    \item Model validation by tests
    \item Macro use of the model for forecasting and policy analysis
\end{enumerate}

Bootstrap on residuals is valid if the residuals are white noise and the process is stationary. \\

ARDL: $Y_t = \alpha + \sum_{j=1}^m \beta_j X_{t-j} + \sum_{j=1}^m \gamma_j Y_{t-j} + \varepsilon_t$. 

\textbf{The model specification is generally carried out using
information criteria. }
\\

About Structural VAR: Structural shocks are supposed to be white noise processes and orthogonal to each others.

we could use short-run restrictions with Cholesky decomposition, but also Local Projection à la Jordà (2005) or sign (long-run) restrictions à la Uhlig (2005).

\subsection{Non-linear regression}
Mention the possible use of a Markov Switching process to account for IRC (up or down), or a price effect (low growth or strong growth) — use Catherine Doz's course for theoretical background.

What is shown by Lantz: Gauss-Newton method (Taylor linearization before OLS), or polynomial regression by minimising Mallows $C_p$ to know the polynomial order. 

\subsection{Forecast (see p.115)}
Boostrapping is useful when the error terms are non-normal. "L’application des méthodes de bootstrap sur les modèles de régression permet d’approximer la distribution des erreurs de prédiction par leur distribution empirique lorsque celle-ci est inconnue. Le bootstrap est ainsi particulièrement utile lorsque les
échantillons de données sont de petite taille et qu’il n’est pas possible de postuler que les erreurs ont une distribution gaussienne"

Guess for a growth rate with `predict` and add fluctuations based on the regression slope and the IRC fluctuation? \\

\begin{itemize}
    \item Estimation MCO du modèle
    \item Prévision MCO
    \item Initialisation de la boucle bootstrap
    \item Boucle bootstrap
    \item Construction de l’intervalle de prédiction bootstrap
\end{itemize}

Les mesures d’erreur quadratique moyenne et d’erreur moyenne absolue permettent de mesurer l’écart entre les prévisions et les observations lorsqu’on effectue
des prévisions sur des données rétrospectives. Les indicateurs obtenus à partir de la statistique U de Theil sont utilisés sur des prévisions rétrospectives afin d’évaluer si les erreurs de prévisions retranscrivent un effet de biais, de variance ou, de préférence, un
effet de covariance.