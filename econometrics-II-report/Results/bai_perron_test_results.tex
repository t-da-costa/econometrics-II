\begin{table}[h]
\centering
\caption{Bai-Perron Sequential Test Results}
\begin{tabular}{lccc}
\hline
Test & Statistic & Critical Value & p-value \\
\hline
The number of breaks is estimated by specified number of breaks 
Partial change model with 2 estimated breaks.
Minimum SSR = 0.000 

Estimated date:
        Break1  Break2
Date        10      18
95% CI (10,12) (17,19)
90% CI (10,12) (18,19)

Estimated regime-specific coefficients:
                          Regime 1      Regime 2      Regime 3
Const (SE)           0.000 (0.000) 0.000 (0.000) 0.000 (0.000)
ln_price2015_MC (SE) 0.000 (0.000) 0.000 (0.000) 0.000 (0.000)
ln_Inc_pc_MC (SE)    0.000 (0.000) 0.000 (0.000) 0.000 (0.000)

Estimated full-sample coefficients:

                 IRC (SE)
Full sample 0.000 (0.000)

The number of breaks is estimated by specified number of breaks 
Partial change model with 2 estimated breaks.
Minimum SSR = 0.000 

Estimated date:
        Break1  Break2
Date        10      18
95% CI (10,12) (17,19)
90% CI (10,12) (18,19)

Estimated regime-specific coefficients:
                          Regime 1      Regime 2      Regime 3
Const (SE)           0.000 (0.000) 0.000 (0.000) 0.000 (0.000)
ln_price2015_MC (SE) 0.000 (0.000) 0.000 (0.000) 0.000 (0.000)
ln_Inc_pc_MC (SE)    0.000 (0.000) 0.000 (0.000) 0.000 (0.000)

Estimated full-sample coefficients:

                 IRC (SE)
Full sample 0.000 (0.000)

The number of breaks is estimated by specified number of breaks 
Partial change model with 2 estimated breaks.
Minimum SSR = 0.000 

Estimated date:
        Break1  Break2
Date        10      18
95% CI (10,12) (17,19)
90% CI (10,12) (18,19)

Estimated regime-specific coefficients:
                          Regime 1      Regime 2      Regime 3
Const (SE)           0.000 (0.000) 0.000 (0.000) 0.000 (0.000)
ln_price2015_MC (SE) 0.000 (0.000) 0.000 (0.000) 0.000 (0.000)
ln_Inc_pc_MC (SE)    0.000 (0.000) 0.000 (0.000) 0.000 (0.000)

Estimated full-sample coefficients:

                 IRC (SE)
Full sample 0.000 (0.000)

The number of breaks is estimated by specified number of breaks 
Partial change model with 2 estimated breaks.
Minimum SSR = 0.000 

Estimated date:
        Break1  Break2
Date        10      18
95% CI (10,12) (17,19)
90% CI (10,12) (18,19)

Estimated regime-specific coefficients:
                          Regime 1      Regime 2      Regime 3
Const (SE)           0.000 (0.000) 0.000 (0.000) 0.000 (0.000)
ln_price2015_MC (SE) 0.000 (0.000) 0.000 (0.000) 0.000 (0.000)
ln_Inc_pc_MC (SE)    0.000 (0.000) 0.000 (0.000) 0.000 (0.000)

Estimated full-sample coefficients:

                 IRC (SE)
Full sample 0.000 (0.000)

The number of breaks is estimated by specified number of breaks 
Partial change model with 2 estimated breaks.
Minimum SSR = 0.000 

Estimated date:
        Break1  Break2
Date        10      18
95% CI (10,12) (17,19)
90% CI (10,12) (18,19)

Estimated regime-specific coefficients:
                          Regime 1      Regime 2      Regime 3
Const (SE)           0.000 (0.000) 0.000 (0.000) 0.000 (0.000)
ln_price2015_MC (SE) 0.000 (0.000) 0.000 (0.000) 0.000 (0.000)
ln_Inc_pc_MC (SE)    0.000 (0.000) 0.000 (0.000) 0.000 (0.000)

Estimated full-sample coefficients:

                 IRC (SE)
Full sample 0.000 (0.000)
   Break_Year F_Statistic     P_Value
F1       2007    5.786218 0.001479995

supF(l+1|l) tests using global optimizers under the null

         supF(1|0) supF(2|1) supF(3|2)
Seq supF    34.725    79.941    12.626
10% CV      11.590    13.430    14.430
5% CV       13.470    15.250    16.360
2.5% CV     15.280    17.080    18.100
1% CV       17.600    19.350    20.020
List of 4
 $ supfl : num [1:3, 1] 34.7 79.9 12.6
 $ cv    : num [1:4, 1:3] 11.6 13.5 15.3 17.6 13.4 ...
  ..- attr(*, "dimnames")=List of 2
  .. ..$ : chr [1:4] "10" "5" "2.5" "1"
  .. ..$ : NULL
 $ mbreak: num 3
 $ sfl   :'data.frame':	5 obs. of  3 variables:
  ..$ supF(1|0): chr [1:5] "34.725" "11.590" "13.470" "15.280" ...
  ..$ supF(2|1): chr [1:5] "79.941" "13.430" "15.250" "17.080" ...
  ..$ supF(3|2): chr [1:5] "12.626" "14.430" "16.360" "18.100" ...
 - attr(*, "class")= chr "seqtests"
List of 4
 $ supfl : num [1:3, 1] 34.7 79.9 12.6
 $ cv    : num [1:4, 1:3] 11.6 13.5 15.3 17.6 13.4 ...
  ..- attr(*, "dimnames")=List of 2
  .. ..$ : chr [1:4] "10" "5" "2.5" "1"
  .. ..$ : NULL
 $ mbreak: num 3
 $ sfl   :'data.frame':	5 obs. of  3 variables:
  ..$ supF(1|0): chr [1:5] "34.725" "11.590" "13.470" "15.280" ...
  ..$ supF(2|1): chr [1:5] "79.941" "13.430" "15.250" "17.080" ...
  ..$ supF(3|2): chr [1:5] "12.626" "14.430" "16.360" "18.100" ...
 - attr(*, "class")= chr "seqtests"

supF(l+1|l) tests using global optimizers under the null

         supF(1|0) supF(2|1) supF(3|2)
Seq supF    34.725    79.941    12.626
10% CV      11.590    13.430    14.430
5% CV       13.470    15.250    16.360
2.5% CV     15.280    17.080    18.100
1% CV       17.600    19.350    20.020
List of 4
 $ supfl : num [1:3, 1] 34.7 79.9 12.6
 $ cv    : num [1:4, 1:3] 11.6 13.5 15.3 17.6 13.4 ...
  ..- attr(*, "dimnames")=List of 2
  .. ..$ : chr [1:4] "10" "5" "2.5" "1"
  .. ..$ : NULL
 $ mbreak: num 3
 $ sfl   :'data.frame':	5 obs. of  3 variables:
  ..$ supF(1|0): chr [1:5] "34.725" "11.590" "13.470" "15.280" ...
  ..$ supF(2|1): chr [1:5] "79.941" "13.430" "15.250" "17.080" ...
  ..$ supF(3|2): chr [1:5] "12.626" "14.430" "16.360" "18.100" ...
 - attr(*, "class")= chr "seqtests"

supF(l+1|l) tests using global optimizers under the null

         supF(1|0) supF(2|1) supF(3|2)
Seq supF    34.725    79.941    12.626
10% CV      11.590    13.430    14.430
5% CV       13.470    15.250    16.360
2.5% CV     15.280    17.080    18.100
1% CV       17.600    19.350    20.020

supF(l+1|l) tests using global optimizers under the null

         supF(1|0) supF(2|1) supF(3|2)
Seq supF    34.725    79.941    12.626
10% CV      11.590    13.430    14.430
5% CV       13.470    15.250    16.360
2.5% CV     15.280    17.080    18.100
1% CV       17.600    19.350    20.020

supF(l+1|l) tests using global optimizers under the null

         supF(1|0) supF(2|1) supF(3|2)
Seq supF    34.725    79.941    12.626
10% CV      11.590    13.430    14.430
5% CV       13.470    15.250    16.360
2.5% CV     15.280    17.080    18.100
1% CV       17.600    19.350    20.020

> # ===================================#
> #      M2 EEET - Econometrics 2     #
> #            2024-2025              #
> #         DA COSTA & GUILLA .... [TRUNCATED] 

> library(dplyr)

> library(tidyr)

> library(ggplot2)

> library(cowplot)

> library(car)

> library(carData)

> library(caret)

> library(FactoMineR)

> library(corrplot) # pour la matrice de corrélation

> library(ggcorrplot)

> library(lmtest) # pour Breusch-Pagan

> library(skedastic) # pour White

> library(nortest) # pour Anderson-Darling

> library(olsrr) # pour White

> library(mbreaks) # for Bai-Perron test

> library(strucchange) # for Chow test

> # Save regression results to LaTeX
> library(stargazer)

> library(xtable)

> # %============================%=
> # DATA LOADING ####
> # %============================%=
> # Téléchargement des données dans un dataframe nommé c .... [TRUNCATED] 

-
/
                                                                                                   

/
                                                                                                   

-
                                                                                                   
> # Charger les données de revenu net
> revenu_net <- read.csv("revenu_net.csv", sep = ";")

> # IPC   (INSEE)	         Indice annuel des prix à la consommation - Base 2015 - Ensemble des ménages - France - Ensemble #nolint
> # PIB   (INSEE)	  .... [TRUNCATED] 

> # Convert Date column to numeric in revenu_net
> revenu_net$Date <- as.numeric(revenu_net$Date)

> # Merge revenu_net with celec on the Date column
> celec <- merge(celec, revenu_net, by = "Date", all = TRUE)

> # Renommer les colonnes pour une meilleure clarté
> celec <- celec %>%
+     # IRC = Indice de Rigueur Climatique.
+     rename(
+         IRC = DJU .... [TRUNCATED] 

> # %===========================================
> # Adding new variables
> 
> # Le PIB est en euro constant 2020 mais l'inflation en base 2015 : on d .... [TRUNCATED] 

> summary(celec)
      Date           IPC            PIB2020         Pelec         Population            IRC        
 Min.   :1990   Min.   : 67.40   Min.   :1566   Min.   :109.9   Min.   :56708831   Min.   :0.8311  
 1st Qu.:1998   1st Qu.: 78.08   1st Qu.:1792   1st Qu.:114.1   1st Qu.:58350214   1st Qu.:0.9322  
 Median :2006   Median : 88.61   Median :2149   Median :124.5   Median :61389492   Median :1.0034  
 Mean   :2006   Mean   : 88.12   Mean   :2065   Mean   :133.6   Mean   :61214776   Mean   :0.9995  
 3rd Qu.:2013   3rd Qu.: 99.58   3rd Qu.:2303   3rd Qu.:148.4   3rd Qu.:63938216   3rd Qu.:1.0475  
 Max.   :2021   Max.   :106.45   Max.   :2505   Max.   :193.1   Max.   :65613522   Max.   :1.1943  
   elec_cons        netInc2020       elec_cons_pc          PIB2015       PIB2015_pc       
 Min.   : 96.91   Min.   : 881376   Min.   :1.709e-06   Min.   :1496   Min.   :2.637e-05  
 1st Qu.:122.49   1st Qu.:1011836   1st Qu.:2.101e-06   1st Qu.:1711   1st Qu.:2.932e-05  
 Median :142.46   Median :1239457   Median :2.303e-06   Median :2052   Median :3.315e-05  
 Mean   :138.43   Mean   :1201547   Mean   :2.252e-06   Mean   :1972   Mean   :3.207e-05  
 3rd Qu.:156.87   3rd Qu.:1352051   3rd Qu.:2.421e-06   3rd Qu.:2199   3rd Qu.:3.438e-05  
 Max.   :166.67   Max.   :1501459   Max.   :2.591e-06   Max.   :2392   Max.   :3.669e-05  
   netInc2015      netInc2015_pc       Pelec2015    
 Min.   : 841570   Min.   :0.01484   Min.   :120.5  
 1st Qu.: 966138   1st Qu.:0.01656   1st Qu.:133.6  
 Median :1183478   Median :0.01928   Median :150.9  
 Mean   :1147281   Mean   :0.01864   Mean   :151.9  
 3rd Qu.:1290987   3rd Qu.:0.02030   3rd Qu.:169.0  
 Max.   :1433648   Max.   :0.02185   Max.   :184.9  

> # %===========================================
> # Going base 100 for all variables in 2015
> 
> base_2015 <- celec %>% filter(Date == 2015)

> celec <- celec %>%
+     mutate(
+         PIB2020_base100_2015 = PIB2020 / base_2015$PIB2020 * 100,
+         Pelec_base100_2015 = Pelec / base_201 .... [TRUNCATED] 

> # %=================================================%=
> # SOME DATA VISUALIZING ####
> # %=================================================%=
> 
>  .... [TRUNCATED] 

> # Tracer les données avec les modifications demandées
> ggplot(celec_long, aes(x = Date, y = value, color = variable)) +
+     geom_line(data = subs .... [TRUNCATED] 

> # Enregistrer le plot
> ggsave("econometrics-II-report/Images/data_base100_2015.jpeg", width = 10, height = 6)

> # %=================================================%=
> # CORRELATION MATRIX ####
> # %=================================================%=
> 
> # S .... [TRUNCATED] 

> # Calculer la matrice de corrélation
> cor_matrix <- cor(numeric_vars, use = "complete.obs")

> # Visualiser la matrice de corrélation
> jpeg("econometrics-II-report/Images/correlation_matrix.jpeg", width = 800, height = 600, quality = 100)

> corrplot(cor_matrix, method = "circle", type = "upper", diag = FALSE, tl.cex = 2, tl.col = "black", p.mat = cor_pmat(numeric_vars), sig.level = 0.05 .... [TRUNCATED] 

> dev.off()
RStudioGD 
        2 

> # %=================================================%=
> # REGRESSION MODEL ####
> # %=================================================%=
> 
> celec .... [TRUNCATED] 

> par(mfrow = c(2, 2))

> plot(celec.lm2)

> summary(celec.lm2)

Call:
lm(formula = log(elec_cons_pc) ~ log(netInc2015_pc) + log(Pelec2015) + 
    IRC, data = celec)

Residuals:
     Min       1Q   Median       3Q      Max 
-0.05843 -0.02068  0.00464  0.01743  0.04668 

Coefficients:
                   Estimate Std. Error t value Pr(>|t|)    
(Intercept)        -9.56344    0.21179 -45.154  < 2e-16 ***
log(netInc2015_pc)  0.85916    0.04413  19.468  < 2e-16 ***
log(Pelec2015)     -0.05713    0.03944  -1.449    0.159    
IRC                 0.26850    0.05686   4.723 5.92e-05 ***
---
Signif. codes:  0 ‘***’ 0.001 ‘**’ 0.01 ‘*’ 0.05 ‘.’ 0.1 ‘ ’ 1

Residual standard error: 0.02642 on 28 degrees of freedom
Multiple R-squared:  0.9439,	Adjusted R-squared:  0.9379 
F-statistic: 157.1 on 3 and 28 DF,  p-value: < 2.2e-16


> ### In the previous one, the price effect is not significant when we take into account the disposable income per capita
> 
> celec.lm <- lm(log(elec .... [TRUNCATED] 

> par(mfrow = c(2, 2))

> plot(celec.lm)

> summary(celec.lm)

Call:
lm(formula = log(elec_cons_pc) ~ log(netInc2015_pc) * log(Pelec2015) + 
    IRC, data = celec)

Residuals:
      Min        1Q    Median        3Q       Max 
-0.046294 -0.015386  0.001337  0.010517  0.048521 

Coefficients:
                                   Estimate Std. Error t value Pr(>|t|)    
(Intercept)                       -36.86048    7.85035  -4.695 6.91e-05 ***
log(netInc2015_pc)                 -6.03750    1.98324  -3.044  0.00515 ** 
log(Pelec2015)                      5.27187    1.53253   3.440  0.00191 ** 
IRC                                 0.29344    0.04865   6.032 1.94e-06 ***
log(netInc2015_pc):log(Pelec2015)   1.34751    0.38743   3.478  0.00173 ** 
---
Signif. codes:  0 ‘***’ 0.001 ‘**’ 0.01 ‘*’ 0.05 ‘.’ 0.1 ‘ ’ 1

Residual standard error: 0.02236 on 27 degrees of freedom
Multiple R-squared:  0.9613,	Adjusted R-squared:  0.9555 
F-statistic: 167.5 on 4 and 27 DF,  p-value: < 2.2e-16


> # Save the bad regression model
> stargazer(celec.lm2, type = "latex", out = "econometrics-II-report/Results/celec_lm2_results.tex")

% Table created by stargazer v.5.2.3 by Marek Hlavac, Social Policy Institute. E-mail: marek.hlavac at gmail.com
% Date and time: Sat, Feb 01, 2025 - 02:57:13
\begin{table}[!htbp] \centering 
  \caption{} 
  \label{} 
\begin{tabular}{@{\extracolsep{5pt}}lc} 
\\[-1.8ex]\hline 
\hline \\[-1.8ex] 
 & \multicolumn{1}{c}{\textit{Dependent variable:}} \\ 
\cline{2-2} 
\\[-1.8ex] & log(elec\_cons\_pc) \\ 
\hline \\[-1.8ex] 
 log(netInc2015\_pc) & 0.859$^{***}$ \\ 
  & (0.044) \\ 
  & \\ 
 log(Pelec2015) & $-$0.057 \\ 
  & (0.039) \\ 
  & \\ 
 IRC & 0.269$^{***}$ \\ 
  & (0.057) \\ 
  & \\ 
 Constant & $-$9.563$^{***}$ \\ 
  & (0.212) \\ 
  & \\ 
\hline \\[-1.8ex] 
Observations & 32 \\ 
R$^{2}$ & 0.944 \\ 
Adjusted R$^{2}$ & 0.938 \\ 
Residual Std. Error & 0.026 (df = 28) \\ 
F Statistic & 157.060$^{***}$ (df = 3; 28) \\ 
\hline 
\hline \\[-1.8ex] 
\textit{Note:}  & \multicolumn{1}{r}{$^{*}$p$<$0.1; $^{**}$p$<$0.05; $^{***}$p$<$0.01} \\ 
\end{tabular} 
\end{table} 

> # Save the good regression model
> stargazer(celec.lm, type = "latex", out = "econometrics-II-report/Results/celec_lm_results.tex")

% Table created by stargazer v.5.2.3 by Marek Hlavac, Social Policy Institute. E-mail: marek.hlavac at gmail.com
% Date and time: Sat, Feb 01, 2025 - 02:57:15
\begin{table}[!htbp] \centering 
  \caption{} 
  \label{} 
\begin{tabular}{@{\extracolsep{5pt}}lc} 
\\[-1.8ex]\hline 
\hline \\[-1.8ex] 
 & \multicolumn{1}{c}{\textit{Dependent variable:}} \\ 
\cline{2-2} 
\\[-1.8ex] & log(elec\_cons\_pc) \\ 
\hline \\[-1.8ex] 
 log(netInc2015\_pc) & $-$6.037$^{***}$ \\ 
  & (1.983) \\ 
  & \\ 
 log(Pelec2015) & 5.272$^{***}$ \\ 
  & (1.533) \\ 
  & \\ 
 IRC & 0.293$^{***}$ \\ 
  & (0.049) \\ 
  & \\ 
 log(netInc2015\_pc):log(Pelec2015) & 1.348$^{***}$ \\ 
  & (0.387) \\ 
  & \\ 
 Constant & $-$36.860$^{***}$ \\ 
  & (7.850) \\ 
  & \\ 
\hline \\[-1.8ex] 
Observations & 32 \\ 
R$^{2}$ & 0.961 \\ 
Adjusted R$^{2}$ & 0.956 \\ 
Residual Std. Error & 0.022 (df = 27) \\ 
F Statistic & 167.504$^{***}$ (df = 4; 27) \\ 
\hline 
\hline \\[-1.8ex] 
\textit{Note:}  & \multicolumn{1}{r}{$^{*}$p$<$0.1; $^{**}$p$<$0.05; $^{***}$p$<$0.01} \\ 
\end{tabular} 
\end{table} 

> # %=================================================%=
> # Tests on the regression ####
> # %=================================================%=
> 
 .... [TRUNCATED] 
               log(netInc2015_pc)                    log(Pelec2015) 
                      3613.788856                       2480.954939 
                              IRC log(netInc2015_pc):log(Pelec2015) 
                         1.177887                       8073.261294 

> ## STRONG MULTICOLLINEARITY
> 
> ## Need for centering the variables
> celec <- celec %>%
+     mutate(
+         ln_Inc_pc_MC = log(netInc2015_pc)  .... [TRUNCATED] 

> # Run regression again with centered variables
> celec.lm <- lm(log(elec_cons_pc) ~ ln_Inc_pc_MC * ln_price2015_MC + IRC, data = celec)

> par(mfrow = c(2, 2))

> plot(celec.lm)

> summary(celec.lm)

Call:
lm(formula = log(elec_cons_pc) ~ ln_Inc_pc_MC * ln_price2015_MC + 
    IRC, data = celec)

Residuals:
      Min        1Q    Median        3Q       Max 
-0.046294 -0.015386  0.001337  0.010517  0.048521 

Coefficients:
                              Estimate Std. Error  t value Pr(>|t|)    
(Intercept)                  -13.29546    0.04853 -273.953  < 2e-16 ***
ln_Inc_pc_MC                   0.72009    0.05471   13.161 2.92e-13 ***
ln_price2015_MC               -0.10374    0.03597   -2.884  0.00761 ** 
IRC                            0.29344    0.04865    6.032 1.94e-06 ***
ln_Inc_pc_MC:ln_price2015_MC   1.34751    0.38743    3.478  0.00173 ** 
---
Signif. codes:  0 ‘***’ 0.001 ‘**’ 0.01 ‘*’ 0.05 ‘.’ 0.1 ‘ ’ 1

Residual standard error: 0.02236 on 27 degrees of freedom
Multiple R-squared:  0.9613,	Adjusted R-squared:  0.9555 
F-statistic: 167.5 on 4 and 27 DF,  p-value: < 2.2e-16


> # Save the good regression model -corrected
> stargazer(celec.lm, type = "latex", out = "econometrics-II-report/Results/celec_lm_results.tex")

% Table created by stargazer v.5.2.3 by Marek Hlavac, Social Policy Institute. E-mail: marek.hlavac at gmail.com
% Date and time: Sat, Feb 01, 2025 - 02:57:15
\begin{table}[!htbp] \centering 
  \caption{} 
  \label{} 
\begin{tabular}{@{\extracolsep{5pt}}lc} 
\\[-1.8ex]\hline 
\hline \\[-1.8ex] 
 & \multicolumn{1}{c}{\textit{Dependent variable:}} \\ 
\cline{2-2} 
\\[-1.8ex] & log(elec\_cons\_pc) \\ 
\hline \\[-1.8ex] 
 ln\_Inc\_pc\_MC & 0.720$^{***}$ \\ 
  & (0.055) \\ 
  & \\ 
 ln\_price2015\_MC & $-$0.104$^{***}$ \\ 
  & (0.036) \\ 
  & \\ 
 IRC & 0.293$^{***}$ \\ 
  & (0.049) \\ 
  & \\ 
 ln\_Inc\_pc\_MC:ln\_price2015\_MC & 1.348$^{***}$ \\ 
  & (0.387) \\ 
  & \\ 
 Constant & $-$13.295$^{***}$ \\ 
  & (0.049) \\ 
  & \\ 
\hline \\[-1.8ex] 
Observations & 32 \\ 
R$^{2}$ & 0.961 \\ 
Adjusted R$^{2}$ & 0.956 \\ 
Residual Std. Error & 0.022 (df = 27) \\ 
F Statistic & 167.504$^{***}$ (df = 4; 27) \\ 
\hline 
\hline \\[-1.8ex] 
\textit{Note:}  & \multicolumn{1}{r}{$^{*}$p$<$0.1; $^{**}$p$<$0.05; $^{***}$p$<$0.01} \\ 
\end{tabular} 
\end{table} 

> # Check VIF again
> vif(celec.lm)
                ln_Inc_pc_MC              ln_price2015_MC                          IRC 
                    2.750454                     1.366358                     1.177887 
ln_Inc_pc_MC:ln_price2015_MC 
                    2.465267 

> # %=================================================%
> # t-test for zero mean
> t.test(residuals(celec.lm), mu = 0)

	One Sample t-test

data:  residuals(celec.lm)
t = 2.9391e-16, df = 31, p-value = 1
alternative hypothesis: true mean is not equal to 0
95 percent confidence interval:
 -0.007523447  0.007523447
sample estimates:
   mean of x 
1.084202e-18 


> # %=================================================%
> # Anderson-Darling test for normality of residuals
> ad.test(residuals(celec.lm))

	Anderson-Darling normality test

data:  residuals(celec.lm)
A = 0.42464, p-value = 0.2989


> # %=================================================%
> # Breusch-Pagan test for heteroskedasticity
> bptest(celec.lm)

	studentized Breusch-Pagan test

data:  celec.lm
BP = 1.9059, df = 4, p-value = 0.7531


> # %=================================================%
> # Goldfeld-Quandt test for heteroskedasticity
> gqtest(celec.lm)

	Goldfeld-Quandt test

data:  celec.lm
GQ = 0.72154, df1 = 11, df2 = 11, p-value = 0.7012
alternative hypothesis: variance increases from segment 1 to 2


> # %=================================================%
> # Durbin-Watson test for autocorrelation
> dw_test <- dwtest(celec.lm)

> # %=================================================%
> # Ljung-Box test for autocorrelation
> ljung_box_test <- Box.test(celec.lm$residuals, lag =  .... [TRUNCATED] 

> # %=================================================%
> # Breusch-Godfrey test for autocorrelation
> bg_test <- bgtest(celec.lm)

> # Export the results to LaTeX
> 
> # Export to LaTeX
> sink("econometrics-II-report/Results/autocorrelation_tests.tex")

> # %=================================================%=
> # PLOT INTERACTION TERMS ####
> # %=================================================%=
> 
> .... [TRUNCATED] 

> # Convert data to long format for ggplot
> interaction_long <- celec %>%
+     select(Date, ln_Inc_pc_MC, ln_price2015_MC, Interaction_ln_price_Inc) .... [TRUNCATED] 

> # Plot the interaction terms
> ggplot(interaction_long, aes(x = Date, y = value, color = variable)) +
+     geom_line(size = 1) +
+     labs(
+      .... [TRUNCATED] 

> # Save the plot
> ggsave("econometrics-II-report/Images/interaction_terms.jpeg", width = 10, height = 6)

> # %=================================================%=
> # WHERE IS THE STRUCTURAL BREAK ? ####
> # %=============================================== .... [TRUNCATED] 

> # Print test results
> print(seq_test_results)

supF(l+1|l) tests using global optimizers under the null

         supF(1|0) supF(2|1) supF(3|2)
Seq supF    34.725    79.941    12.626
10% CV      11.590    13.430    14.430
5% CV       13.470    15.250    16.360
2.5% CV     15.280    17.080    18.100
1% CV       17.600    19.350    20.020

> ## Where are the breaks?
> # Estimate breakpoints with 2 breaks
> break_model <- dofix(
+     y_name = "elec_cons_pc",
+     z_name = c("ln_price201 ..." ... [TRUNCATED] 

> # View estimated break dates
> print(break_model)

The number of breaks is estimated by specified number of breaks 
Partial change model with 2 estimated breaks.
Minimum SSR = 0.000 

Estimated date:
        Break1  Break2
Date        10      18
95% CI (10,12) (17,19)
90% CI (10,12) (18,19)

Estimated regime-specific coefficients:
                          Regime 1      Regime 2      Regime 3
Const (SE)           0.000 (0.000) 0.000 (0.000) 0.000 (0.000)
ln_price2015_MC (SE) 0.000 (0.000) 0.000 (0.000) 0.000 (0.000)
ln_Inc_pc_MC (SE)    0.000 (0.000) 0.000 (0.000) 0.000 (0.000)

Estimated full-sample coefficients:

                 IRC (SE)
Full sample 0.000 (0.000)

> plot_model(break_model, title = "Structural Breaks in Electricity Consumption") +
+     scale_x_continuous(breaks = seq(1, 32, by = 5), labels = cel .... [TRUNCATED] 

> # Save the plot for structural breaks
> ggsave("econometrics-II-report/Images/structural_breaks.jpeg", width = 10, height = 6)

> # %===========================================
> # Estimate structural breaks using Chow test
> index_2006 <- which(celec$Date == 2006)

> index_2011 <- which(celec$Date == 2011)

> years <- index_2006:index_2011

> chow_results <- data.frame(Break_Year = integer(), F_Statistic = numeric(), P_Value = numeric())

> n <- length(celec$elec_cons_pc)

> X <- matrix(c(celec$ln_Inc_pc_MC, celec$ln_price2015_MC, celec$Interaction_ln_price_Inc, celec$IRC), ncol = 4)

> Y <- matrix(celec$elec_cons_pc, n, 1)

> #
> # transformation des variables en logarithme
> y <- log(Y)

> # x <- log(X)
> # x <- cbind(x, c(celec$IRC, log(celec$netInc2015_pc) * log(celec$Pelec2015)))
> 
> for (i in years) {
+     chow_test <- sctest(y ~ .... [TRUNCATED] 

> best_break <- chow_results[which.min(chow_results$P_Value), ]

> print(best_break)
   Break_Year F_Statistic     P_Value
F1       2007    5.786218 0.001479995

> ggplot(chow_results, aes(x = Break_Year, y = F_Statistic)) +
+     geom_line() +
+     geom_point() +
+     geom_hline(yintercept = qf(0.95, df1 = 3 .... [TRUNCATED] 

> # Save the plot
> ggsave("econometrics-II-report/Images/chow_test_results.jpeg", width = 10, height = 6)

> # # Apply the CUSUM-square test
> # cusum_sq_test <- efp(celec.lm$residuals ~ 1, type = "Rec-CUSUM")
> 
> # # Plot without x-axis labels
> # plot(cu .... [TRUNCATED] 

> celec_after_2007 <- celec %>% filter(Date > 2007)

> # Perform regression on the subset
> celec_before_2007.lm <- lm(log(elec_cons_pc) ~ log(ln_Inc_pc_MC)*log(ln_price2015_MC) + IRC, data = celec_befor .... [TRUNCATED] 

The number of breaks is estimated by specified number of breaks 
Partial change model with 2 estimated breaks.
Minimum SSR = 0.000 

Estimated date:
        Break1  Break2
Date        10      18
95% CI (10,12) (17,19)
90% CI (10,12) (18,19)

Estimated regime-specific coefficients:
                          Regime 1      Regime 2      Regime 3
Const (SE)           0.000 (0.000) 0.000 (0.000) 0.000 (0.000)
ln_price2015_MC (SE) 0.000 (0.000) 0.000 (0.000) 0.000 (0.000)
ln_Inc_pc_MC (SE)    0.000 (0.000) 0.000 (0.000) 0.000 (0.000)

Estimated full-sample coefficients:

                 IRC (SE)
Full sample 0.000 (0.000)

supF(l+1|l) tests using global optimizers under the null

         supF(1|0) supF(2|1) supF(3|2)
Seq supF    34.725    79.941    12.626
10% CV      11.590    13.430    14.430
5% CV       13.470    15.250    16.360
2.5% CV     15.280    17.080    18.100
1% CV       17.600    19.350    20.020

supF(l+1|l) tests using global optimizers under the null

         supF(1|0) supF(2|1) supF(3|2)
Seq supF    34.725    79.941    12.626
10% CV      11.590    13.430    14.430
5% CV       13.470    15.250    16.360
2.5% CV     15.280    17.080    18.100
1% CV       17.600    19.350    20.020
